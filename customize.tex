% input this file: \input{/Users/he/Documents/LaTeX/Notebooks/notebook_customize.tex}
% documentclass need: ctex related (ctexart, ctexbook, ...)
% test

% global setting
    \ctexset{autoindent = 0em} % cancel autoindent

    % keep appropriate distance while there being a large equation in a row
    \setlength\lineskiplimit{0.3em} 
    \setlength\lineskip{0.3em}

    % change the symbol size in math mode
    \DeclareMathSizes{10.54}{10.54}{5.27}{3.16}


% other packages
%    \usepackage{pdfsync}
    \usepackage{import}
    \usepackage{xifthen}
    \usepackage{pdfpages}
    \usepackage{transparent}

    % package in drawing
    \usepackage{tikz}
    \usetikzlibrary{optics} % use library: optics (made the draw of optic much easier)
    \usetikzlibrary{arrows.meta} % more type of arrow
    \usetikzlibrary{intersections} % calcuate intersection
    \usetikzlibrary{calc}

    % package using in text mode
    \usepackage[a4paper, left=2.54cm, right=2.54cm, top=2.54cm, bottom=2.54cm]{geometry}
    \usepackage{listings}
    \usepackage{xcolor}
    \usepackage{enumitem}
    \usepackage{graphicx}
    \usepackage{float}
    \usepackage{booktabs}
    \usepackage{caption} % control the caption format of picture
    \usepackage{longtable} % auto-change-page-table
    % \usepackage{hyperref} % provide command: \autoref
    \usepackage{nameref} % provide command: \nameref
    \usepackage{fontspec} % font select and set
    \usepackage{CJKfntef} % enhanced under modify, the parameter can cancel replacing '\emph' with '\CJKunderline'
    \usepackage[normalem]{ulem} % 除可以用于下划线修饰以外, 还提供了指令 \bgroup 与 \markoverwith \Ulon, 可用于自定义文字上的风格.

    % package using in math mode
    \usepackage{amsmath}
        \allowdisplaybreaks[4] % allowed multiline equation can change page
        % \everymath{\displaystyle} % set default math style as displaystyle
    \usepackage{amssymb}
    \usepackage{mathtools}
    \usepackage{mathrsfs} % provide the font of \mathscr
    \usepackage{tensor}
    \usepackage{upgreek}
    \usepackage{extarrows}
    \usepackage{yhmath}
    \usepackage{nccmath} % realign the equation
    \usepackage{xfrac} % split level fractions
    \usepackage{bbm} % hollow number
    \usepackage{bm} % bold, but keep italic

% self define
    \definecolor{CodeGrey}{RGB}{240, 240, 240} % the grey used in code block or single code
    \definecolor{InLineCodeColor}{RGB}{94, 160, 40} % the grey used in code block or single code

% command/environment building and renewing
    % text
        % env
            % Solve: used to "proof", "solve" and so on
            \newenvironment{Solve}
            [1][解:]%
            {\paragraph*{#1\\}}%
            {\begin{flushright}$\square$\end{flushright}}

        % command
                \newcommand\hl{\bgroup\markoverwith{\textcolor{CodeGrey}{\rule[-.8ex]{2pt}{2.8ex}}}\ULon}
                \newcommand\CodeBg{\bgroup\markoverwith{\textcolor{CodeGrey}{\ttfamily\rule[-.8ex]{2pt}{2.8ex}}}\ULon} % code, had the less height "ex", which means even the punctuation like comma had the height "ex".
            \newcommand{\Code}[1]{\lstinline[basicstyle=\color{InLineCodeColor}\ttfamily]{#1}} % in-line code
            \newcommand{\CodeB}[1]{\colorbox{CodeGrey}{\rule{0pt}{1ex}\texttt{#1}}} % code with box, had the less height "ex", which means even the punctuation like comma had the height "ex".
            \newcommand{\completesolving}{\begin{flushright}$\square$\end{flushright}} % the square
            \newcommand{\CompleteExample}{\begin{flushright}$\blacksquare$\end{flushright}} % the black square
            \newcommand\DoubleQuote[1]{``#1''} % double quote
            \newcommand\SingleQuote[1]{`#1'} % single quote

        % font
            \newfontfamily\Zapfino{Zapfino}

    % math
        % command
            \newcommand{\FDEq}[1]{\fbox{$\displaystyle #1$}} % framed display-style equation
        % notation
            \newcommand{\Dbar}{\mathrm{d} \hspace*{-0.15em}\bar{}\hspace*{0.1em}} % \mathrm{d} with bar
            \newcommand\MaE{\mspace{2mu}\mathrm{e}\mspace{2mu}} % "Ma" is the abbreviation of "Math", "E" means natural constant "e"
            \newcommand\MaPI{\mspace{2mu}\uppi\mspace{2mu}} % ratio of the circumference of a circle to its diameter

    %set the display style of enumerate number
    \renewcommand\theenumi{\arabic{enumi}}
    \renewcommand\labelenumi{\theenumi).}
    \renewcommand\theenumii{\arabic{enumii}}
    \renewcommand\theenumiii{\Alph{enumiii}}
    \renewcommand\theenumiv{\Roman{enumiv}}

    % theorem environment
    \makeatletter
    \@ifclassloaded{ctexbook}{%
        \newtheorem{Claim}{Claim}[chapter]%
        \newtheorem{Concept}{Concept}[chapter]%
        \newtheorem{Definition}{Definition}[chapter]%
        \newtheorem{Example}{Example}[chapter]%
        \newtheorem{Formula}{Formula}[chapter]%
        \newtheorem{Law}{Law}[chapter]%
        \newtheorem{Property}{Property}[chapter]%
        \newtheorem{Theorem}{Theorem}[chapter]%
    }{}
    \makeatother

% initialize the setting
    % avoid too large interval between picture's caption and picture
    \captionsetup{skip = 5pt}

    \setcounter{secnumdepth}{5}
    \ctexset{
        section/name = \S\,,
        subsection/beforeskip = 1.3ex plus 0.4ex minus 0.08ex, % 40% of original
        subsection/afterskip = 0.375ex plus 0.05ex, % 25% of original
        paragraph/aftertitle = \hspace{-0.125em},
        paragraph/aftername = \hspace{0.3em},
        paragraph/number = \arabic{paragraph}.,
        paragraph/beforeskip = 0.4875ex plus 0.15ex minus 0.03ex, % 15% of original
        paragraph/afterskip = 0.15em, % 15% of original
        paragraph/hang = false
    }

    \lstset{
    tabsize=4, % size of tab
    xleftmargin=2em, % distance between frame and paper edge
    xrightmargin=2em,
    framexleftmargin=1.5em, % are the dimensions which are used additionally to framesep to make up the margin of a     frame (distance between code and frame)
    framexrightmargin=1.5em, 
    basicstyle=\ttfamily,
    breaklines=true,
    columns=flexible,
    numbers=left,                                        % 在左侧显示行号
    numberstyle=\color{gray},                       % 设定行号格式
    numbersep=5pt,
    frame=none,                                          % 不显示背景边框
    backgroundcolor=\color{CodeGrey},               % 设定背景颜色
    keywordstyle=\color[RGB]{40,40,255},                 % 设定关键字颜色
    numberstyle=\footnotesize\color{darkgray},
    commentstyle=\ttfamily\color[RGB]{49,150,49},                % 设置代码注释的格式
    stringstyle=\rmfamily\slshape\color[RGB]{128,0,0},   % 设置字符串格式
    showstringspaces=false                               % 不显示字符串中的空格
    }

    \setlist[enumerate]{
        itemsep = 0pt,
        topsep = 0pt,
        partopsep = 0pt,
        parsep = 0pt,
        labelsep = 0.3em,
    }
